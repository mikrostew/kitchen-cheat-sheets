% main ratios document
\documentclass[letterpaper,11pt]{article}
\usepackage[margin=1in]{geometry}

\usepackage{tabularx}
\usepackage[table]{xcolor}

\usepackage[tracking=smallcaps]{microtype}
\raggedright

\usepackage{enumitem}

% \usepackage{fancyhdr}
% \pagestyle{fancy}
% \fancyhf{}
% \lhead{ \huge\scshape{ratios} }
% \renewcommand{\headrulewidth}{0pt}
% \renewcommand{\footrulewidth}{0pt}

% section headers - small-caps title
\newcommand{\sectionheader}[1]{
    \vspace{0.5em}
    {\small\textsc{#1}} \\
}


\begin{document}
\normalsize

\pagestyle{empty}
{\huge\scshape{ratios}}

\rowcolors{2}{gray!25}{white}

\sectionheader{doughs}
{ \small
\begin{tabularx}{\textwidth}{ X X X X X X }
\rowcolor{white}
% ~ & flour & fat & liquid & egg & sugar \\
~ \\
% \hline
Bread & 5 parts flour & & 3 parts water & & \\
Pasta Dough & 3 parts flour & & & 2 parts egg & \\
Pie Dough & 3 parts flour & 2 parts fat & 1 part water & & \\
Biscuit & 3 parts flour & 1 part fat & 2 parts liquid & & \\
Cookie Dough & 3 parts flour & 2 parts fat & & & 1 part sugar \\
Pate a Choux & 1 part flour & 1 part butter & 2 parts water & 2 parts egg & \\
\end{tabularx}
}

\sectionheader{batters}
{ \small
\begin{tabularx}{\textwidth}{ l X X X X X }
\rowcolor{white}
% ~ & flour & sugar & butter & egg & liquid \\
~ \\
% \hline
Pound Cake & 1 part flour & 1 part sugar & 1 part butter & 1 part egg & \\
Sponge Cake & 1 part flour & 1 part sugar & 1 part butter & 1 part egg & \\
Angel Food Cake & 1 part flour & 3 parts sugar & & 3 parts white & \\
Quick Bread & 2 parts flour & & 1 part butter & 1 part egg & 2 parts liquid \\
Muffin & 2 parts flour & & 1 part butter & 1 part egg & 2 parts liquid \\
Fritter & 2 parts flour & & & 1 part egg & 2 parts liquid \\
Pancake & 2 parts flour & & $\frac{1}{2}$ part butter & 1 part egg & 2 parts liquid \\
Popover & 1 part flour & & & 1 part egg & 2 parts liquid \\
Crepe & $\frac{1}{2}$ part flour & & 1 part butter & 1 part egg & \\
\end{tabularx}
}

\sectionheader{stocks and sauces}
{ \small
\begin{tabularx}{\textwidth}{ l X X X X }
\rowcolor{white}
~ \\
% \hline
Stock & 3 parts water & 2 parts bones & & \\
Consomme & 12 parts stock & 3 parts meat & 1 part mirepoix & 1 part egg white \\
Roux & 3 parts flour & 2 parts fat & & \\
Thickening Ratio & 10 parts liquid & 1 part roux & & \\
Beurre Manie & 1 part flour & 1 part butter & (by volume) & \\
Slurry & 1 part cornstarch & 1 part water & (by volume) & \\
\end{tabularx}
}

{ \small
\begin{itemize}[topsep=4pt]
  \item Thickening rule: 1 Tbsp starch thickens 1 cup liquid
\end{itemize}
}

\sectionheader{meat}
{ \small
\begin{tabularx}{\textwidth}{ l X X X }
\rowcolor{white}
~ \\
% \hline
Sausage & 3 parts meat & 1 part fat & \\
Sausage Seasoning & 60 parts meat/fat & 1 part salt \\
Mousseline & 8 parts meat & 4 parts cream & 1 part egg \\
Brine & 20 parts water & 1 part salt & \\
\end{tabularx}
}

\sectionheader{fat-based sauces}
{ \small
\begin{tabularx}{\textwidth}{ l X X X }
\rowcolor{white}
~ \\
% \hline
Mayonnaise & 20 parts oil & 1 part liquid (plus yolk) & \\
Vinaigrette & 3 parts oil & 1 part vinegar & \\
Hollandaise & 5 parts butter & 1 part yolk & 1 part liquid \\
\end{tabularx}
}

\sectionheader{custards}
{ \small
\begin{tabularx}{\textwidth}{ l X X X }
\rowcolor{white}
~ \\
% \hline
Free-Standing Custard & 2 parts liquid & 1 part egg & \\
Creme Anglaise & 4 parts milk/cream & 1 part yolk & 1 part sugar \\
Chocolate Sauce & 1 part chocolate & 1 part cream & \\
Caramel Sauce & 1 part sugar & 1 part cream & \\
\end{tabularx}
}


\end{document}
