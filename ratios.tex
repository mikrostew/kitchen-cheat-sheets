% main ratios document
\documentclass[letterpaper,11pt]{article}
\usepackage[margin=1in]{geometry}

% \usepackage{setspace}
% \usepackage{enumitem}
% \usepackage{etoolbox}
\usepackage[table]{xcolor}

%\usepackage{fontawesome}
\usepackage[tracking=smallcaps]{microtype}
\raggedright

% section headers - small-caps title
\newcommand{\sectionheader}[1]{
    \vspace{1.2em}
    {\small\textsc{#1}} \\
}


\begin{document}
\pagestyle{empty}

\normalsize

{\huge\scshape{ratios}}

\rowcolors{2}{gray!25}{white}

\sectionheader{doughs}
\begin{tabular}{ l c c c c c }
\rowcolor{white}
~ & flour & fat & liquid & egg & sugar \\
\hline
Bread & 5 parts & & 3 parts water & & \\
Pasta Dough & 3 parts & & & 2 parts egg & \\
Pie Dough & 3 parts & 2 parts fat & 1 part water & & \\
Biscuit & 3 parts & 1 part fat & 2 parts liquid & & \\
Cookie Dough & 3 parts & 2 parts fat & & & 1 part sugar \\
Pate a Choux & 1 part & 1 part butter & 2 parts water & 2 parts egg & \\
\end{tabular}

\sectionheader{batters}
\begin{tabular}{ l c c c c c }
\rowcolor{white}
~ & flour & sugar & butter & egg & liquid \\
\hline
Pound Cake & 1 part & 1 part & 1 part & 1 part & \\
Sponge Cake & 1 part & 1 part & 1 part & 1 part & \\
Angel Food Cake & 1 part & 3 parts & & 3 parts (white) & \\
Quick Bread & 2 parts & & 1 part & 1 part & 2 parts \\
Muffin & 2 parts & & 1 part & 1 part & 2 parts \\
Fritter & 2 parts & & & 1 part & 2 parts \\
Pancake & 2 parts & & $\frac{1}{2}$ part & 1 part & 2 parts \\
Popover & 1 part & & & 1 part & 2 parts \\
Crepe & $\frac{1}{2}$ part & & 1 part & 1 part \\
\end{tabular}

\sectionheader{stocks and sauces}
\begin{tabular}{ l c c c c }
\rowcolor{white}
~ \\
\hline
Stock & 3 parts water & 2 parts bones & & \\
Consomme & 12 parts stock & 3 parts meat & 1 part mirepoix & 1 part egg white \\
Roux & 3 parts flour & 2 parts fat & & \\
Thickening Ratio & 10 parts liquid & 1 part roux & & \\
Beurre Manie & 1 part flour & 1 part butter & (by volume) & \\
Slurry & 1 part cornstarch & 1 part water & (by volume) & \\
\end{tabular}

\begin{itemize}
  \item Thickening rule: 1 Tbsp starch thickens 1 cup liquid
\end{itemize}

\sectionheader{meat}
\begin{tabular}{ l c c c }
\rowcolor{white}
~ \\
\hline
Sausage & 3 parts meat & 1 part fat & \\
Sausage Seasoning & 60 parts meat/fat & 1 part salt \\
Mousseline & 8 parts meat & 4 parts cream & 1 part egg \\
Brine & 20 parts water & 1 part salt & \\
\end{tabular}

\sectionheader{fat-based sauces}
\begin{tabular}{ l c c c }
\rowcolor{white}
~ \\
\hline
Mayonnaise & 20 parts oil & 1 part liquid (plus yolk) & \\
Vinaigrette & 3 parts oil & 1 part vinegar & \\
Hollandaise & 5 parts butter & 1 part yolk & 1 part liquid \\
\end{tabular}

\sectionheader{custards}
\begin{tabular}{ l c c c }
\rowcolor{white}
~ \\
\hline
Free-Standing Custard & 2 parts liquid & 1 part egg & \\
Creme Anglaise & 4 parts milk/cream & 1 part yolk & 1 part sugar \\
Chocolate Sauce & 1 part chocolate & 1 part cream & \\
Caramel Sauce & 1 part sugar & 1 part cream & \\
\end{tabular}


\end{document}
