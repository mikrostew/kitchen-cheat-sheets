% main ratios document
\documentclass[letterpaper,11pt]{article}
\usepackage[margin=0.8in]{geometry}

\usepackage{tabularx}
\usepackage[table]{xcolor}

\usepackage[tracking=smallcaps]{microtype}
\raggedright

\usepackage{enumitem}

\usepackage{tikz}

\usepackage{stackengine}

\usepackage{scrextend}


% section headers - small-caps title
\newcommand{\sectionheader}[1]
{
  \vspace{0.5em}
  {\small\textsc{#1}} \\
  \vspace{1em}
}%

% section table layout
\newcommand{\sectiontable}[1]
{
  % \small
  \rowcolors{1}{green!20}{white}
  % rounded corners
  % (see http://tex.stackexchange.com/questions/184060/table-with-alternating-row-colours-and-rounded-corners)
  \noindent\begin{tikzpicture}[line width=.1mm,rounded corners=.5em]
    \node(thetable) [clip,inner sep=.5\pgflinewidth] {
      \addstackgap[\pgflinewidth]{%
      \begin{tabularx}{\dimexpr\textwidth-2\pgflinewidth\relax}{ p{4cm} X c X c X c X }
        #1
      \end{tabularx}%
      }
    };
    \draw ([xshift=.5*\pgflinewidth,yshift=-.5*\pgflinewidth]thetable.north west)
      rectangle ([xshift=-.5*\pgflinewidth,yshift=.5*\pgflinewidth]thetable.south east);
  \end{tikzpicture}
}%

% for notes under a table
\newcommand{\note}[1]
{
  \small
  \begin{addmargin}{1em}
    #1
  \end{addmargin}
}%


\begin{document}
\normalsize

\pagestyle{empty}
{\huge\scshape{conversions}} \\
~ \\

\sectionheader{temperature}

  \noindent\begin{tikzpicture}[line width=.1mm,rounded corners=.5em]
    \node(thetable) [clip,inner sep=.5\pgflinewidth] {
      \addstackgap[\pgflinewidth]{%
\begin{tabular}{ c c c }
\rowcolor{white}
Fahrenheit & Celcius & Gas Mark \\
\hline
\rowcolor{red!5}
225 & 105 & $\frac{1}{4}$ \\
\rowcolor{red!10}
250 & 120 & $\frac{1}{2}$ \\
\rowcolor{red!15}
275 & 135 & 1 \\
\rowcolor{red!20}
300 & 150 & 2 \\
\rowcolor{red!25}
325 & 165 & 3 \\
\rowcolor{red!30}
350 & 180 & 4 \\
\rowcolor{red!35}
375 & 190 & 5 \\
\rowcolor{red!40}
400 & 205 & 6 \\
\rowcolor{red!45}
425 & 220 & 7 \\
\rowcolor{red!50}
450 & 230 & 8 \\
\rowcolor{red!55}
475 & 245 & 9 \\
\rowcolor{red!55}
500 & 260 & 10 \\
\rowcolor{red!55}
525 & 275 & 11 \\
\rowcolor{red!55}
550 & 290 & 12 \\
\end{tabular}%
      }
    };
    \draw ([xshift=.5*\pgflinewidth,yshift=-.5*\pgflinewidth]thetable.north west)
      rectangle ([xshift=-.5*\pgflinewidth,yshift=.5*\pgflinewidth]thetable.south east);
  \end{tikzpicture}

\note{ $ ^{\circ}F ~=~ (^{\circ}C * \frac{9}{5}) + 32 $ }
\note{ $ ^{\circ}C ~=~ (^{\circ}F - 32) * \frac{5}{9} $ }

\sectionheader{weight}
\rowcolors{2}{green!20}{white}
\begin{tabular}{ l l }
\rowcolor{white}
Imperial & Metric \\
\hline
$\frac{1}{2}$ oz & 15g \\
$\frac{3}{4}$ oz & 20g \\
1 oz & 30g \\
2 oz & 60g \\
3 oz & 85g \\
4 oz ($\frac{1}{4}$ lb) & 115g \\
5 oz & 140g \\
6 oz & 170g \\
7 oz & 200g \\
8 oz ($\frac{1}{4}$ lb) & 230g \\
9 oz & 255g \\
10 oz & 285g \\
11 oz & 310g \\
12 oz ($\frac{1}{4}$ lb) & 340g \\
13 oz & 370g \\
14 oz & 400g \\
15 oz & 425g \\
16 oz (1 lb) & 450g \\
24 oz & 680g \\
32 oz (2 lb) & 0.9kg \\
48 oz (3 lb) & 1.4kg \\
64 oz (4 lb) & 1.8kg \\
\end{tabular}%

\note{ $ 1kg ~=~ 35oz (2.2lbs) $ }
\note{ $ 1oz ~=~ 28.35g $ }
\note{ $ 1g ~=~ 0.035oz $ }



\end{document}
